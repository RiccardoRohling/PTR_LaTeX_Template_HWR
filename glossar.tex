% This file defines the glossary and acronyms used in the document.

% Acronyms are defined using the \newacronym command.
% The format is: \newacronym{label}{abbr}{long text}
\newacronym{bi}{BI}{Business Intelligence}
\newacronym{kpi}{KPI}{Key Performance Indicator}
\newacronym{saas}{SaaS}{Software as a Service}
\newacronym{crm}{CRM}{Customer Relationship Management}
\newacronym{erp}{ERP}{Enterprise Resource Planning}
\newacronym{api}{API}{Application Programming Interface}
\newacronym{oas}{OAS}{OpenAPI Specification}
\newacronym{rd}{R\&D}{Research and Development}
\newacronym{dses}{DSES}{Data Solutions Enablement Services}
\newacronym{soap}{SOAP}{Simple Object Access Protocol}
\newacronym{grpc}{gRPC}{ein von Google entwickeltes Remote Procedure Call Protokoll}
\newacronym{rest}{REST}{Representational State Transfer}
\newacronym{http}{HTTP}{Hypertext Transfer Protocol}
\newacronym{rfc}{RFC}{Request for Comments}
\newacronym{ws}{WebSockets}{ein Netzwerkprotokoll}
\newacronym{gq}{GraphQL}{ein von Meta entwickeltes API Protokoll}
%\newacronym[longplural={System Organ Classes}]{SOC}{SOC}{System Organ Class}
\newacronym{KI}{KI}{Künstliche Intelligenz}
\newacronym{KuO}{KuO}{Körperregionen und Organe}
\newacronym{RGB}{RGB}{Rot Grün Blau}
\newacronym{WHO-ART}{WHO-ART}{Adverse Reaction Terminology von der Weltgesundheitsorganisation}
\newacronym{COSTART}{COSTART}{Coding Symbols for a Thesaurus of Adverse Reaction Terms}
\newacronym{ICD}{ICD}{International Classification of Diseases}
\newacronym{MedDRA}{MedDRA}{Medical Dictionary for Regulatory Activities}
\newacronym[longplural={Preferred Terms}]{PT}{PT}{Preferred Term}
\newacronym[longplural={Low Level Terms}]{LLT}{LLT}{Low Level Term}
\newacronym[longplural={High Level Terms}]{HLT}{HLT}{High Level Term}
\newacronym{HLGT}{HLGT}{High Level Group Term}
\newacronym{MVT}{MVT}{Model, View, Template}
\newacronym{JSX}{JSX}{JavaScript XML}
\newacronym{HTML}{HTML}{Hypertext Markup Language}


% Glossary entries are defined using the \newglossaryentry command.
% The format is: \newglossaryentry{label}{name=..., description=...}
% \newglossaryentry{hypermedia}{name=Hyper\-media-Systeme, description={Eine Integration verschiedener Multimedia-Elemente wie Text, Bilder, Audio und Video in ein einziges, miteinander verbundenes System \cite{Hypermedia}.}}
\newglossaryentry{Connected-System}{name=Connected System, description={Ein Werkzeug in der Low-Code Plattform \say{Appian}, das für die Kommunikation mit den externen Systemen eingesetzt wird.}}
\newglossaryentry{Tags}{name=Tags, description={Gruppen, in die die 27 \glspl{SOC} eingeteilt sind}}
\newglossaryentry{Gimp}{name=Gimp, description={pixelbasiertes Bildbearbeitungsprogramm}}
\newglossaryentry{medizinische Wörterbücher}{name=medizinische Wörterbücher, description={System für eine einheitliche Dokumentierung der während einer klinischen Studie auftretenden Nebenwirkungen}}
\newglossaryentry{Django}{name=Django, description={Python Webframework}}
\newglossaryentry{Django REST Framework}{name=Django REST Framework, description={Tool für die API-Entwicklung in Python}}
\newglossaryentry{Python}{name=Python, description={Programmiersprache}}
\newglossaryentry{React}{name=React, description={Bibliothek für die Frontendentwicklung}}
\newglossaryentry{Hook}{name=Hook, description={eine in React eingebettete Funktion, die beispielsweise für das Statemanagement verwendet werden kann}}
\newglossaryentry{JSON}{name=JSON, description={JavaScript Object Notation, kompaktes Datenformat für den Datenaustausch}}
\newglossaryentry{Frontend}{name=Frontend, description={sichtbarer und bedienbarer Teil einer Webseite}}
\newglossaryentry{Backend}{name=Backend, description={unsichtbarer Teil einer Anwendung, der Server, Datenbanken und Logik verwaltet}}
\newglossaryentry{svg}{name=svg, description={Scalable Vector Graphics}}
\newglossaryentry{csv}{name=csv, description={Comma-Separated-Values}}
