\chapter{Introduction}
Research questions:


    \begin{enumerate}[label=\textit{\arabic*.}]
        \item \textit{What key factors influenced Bayer's decision to transition 
        from Tableau to Power BI, particularly in terms of cost, functionality, 
        and organizational needs?}
        \item \textit{How has the transition to Power BI impacted the PCT 
        Team within \gls{rd} Finance, specifically regarding workflow 
        efficiency and data accessibility?}
        \item \textit{What strategies and best practices can the PCT Team 
        implement to successfully migrate dashboards from Tableau to Power BI, 
        ensuring minimal disruption and enhanced performance?}
    \end{enumerate}
    

\chapter{Understanding Business Intelligence and Dashboard Technology}

\gls{bi} encompasses a complex interplay of
processes and products that facilitate effective 
decision-making through the integration and analysis of an 
organization’s data resources \autocite[p.1]{taveraromeroBusinessIntelligenceBusiness2021}. 
This process involves various methodologies that organizations 
adopt to transform raw data into valuable information or 
insights, which is essential for success in today’s 
competitive global landscape \autocite[p.121]{jourdanBusinessIntelligenceAnalysis2008}. 
Moreover, BI empowers businesses to formulate critical 
questions by leveraging data derived from diverse operations, 
customer interactions, and competitive insights, both within 
their industry and beyond. This capability not only aids in 
identifying new growth opportunities but also enhances 
organizational resilience \autocite[p.308]{kumarBusinessIntelligence2024}. 
%%%%%%%%%%%%%%%%%%%%%%%%%%%%%%%%%%%%%
%Maybe consider to have also a critical perspective on it, that BI CAN 
%empower the busniess but challanges are also historical grown data 
%structure / no data structure etc.
%%%%%%%%%%%%%%%%%%%%%%%%%%%%%%%%%%%%%
When comparing these perspectives, it is clear that while 
\citeauthor{taveraromeroBusinessIntelligenceBusiness2021} 
emphasize the technical integration 
and analysis aspect, \citeauthor{jourdanBusinessIntelligenceAnalysis2008} 
focus on the transformative process that converts data into actionable 
intelligence. \citeauthor{kumarBusinessIntelligence2024} 
further expands on this by highlighting 
the proactive role of \gls{bi} in strategic decision-making and 
identifying opportunities. Each perspective contributes to 
a comprehensive understanding of BI, illustrating its 
evolution from a data-centric tool ecosystem to a strategic asset 
that empowers organizations to navigate complex market 
dynamics effectively.

\begin{figure}[h]
    \centering
    \includegraphics[width=12.5cm]{images/BI_Reports.png}
    \caption{Types of BI Reports}
    \label{fig:types_of_bi_reports}
    \begin{flushleft}
        \small{Source: Own Creation}
    \end{flushleft}    
\end{figure}

It continues to evolve as a strategic asset, offering 
diverse types of reports that enhance decision-making 
through the integration and analysis of data resources. 
As seen in Fig.\ref{fig:types_of_bi_reports}, these reports are categorized into descriptive, 
predictive, and prescriptive types, each representing an evolution
of the usage of data resources.
purposes within an organization. 
Descriptive reports focus on analyzing past events using 
\gls{kpi} and employ root cause 
analysis to evaluate reasons behind these events, 
exemplified by operational reporting. Predictive reports 
extend beyond historical data by utilizing algorithms
%%%%%%%%%%%%%%%%%%%%%%%%%%%%%%%%%%%%%
%Is a definition needed before introducing this term?
%%%%%%%%%%%%%%%%%%%%%%%%%%%%%%%%%%%%%
such as neural networks, logistic regressions, and 
linear regressions to forecast near-future trends and option spaces. 
Meanwhile, prescriptive reports provide actionable 
recommendations, suggesting adjustments to specific 
factors within a business model. These factors are 
input into a model that can be refined, acknowledging 
the interdependence between them \autocite[pp. 308-310]{kumarBusinessIntelligence2024}.
 
A central part of \gls{bi} are Dashboards. They are visual
representations of \gls{bi} data that serve as an influential instrument for 
decision-makers by facilitating the monitoring of 
production processes to enhance performance 
\autocite[p. 407]{belghithDataVisualizationIndustry2023}. 
In a case study conducted by Katta (\citeyear[p.68]{kattaLeveragingPowerBI2025}), 
the implementation of the \gls{bi} Platform Power BI within a 
pharmaceutical company was evaluated, revealing significant financial 
savings through the use of dashboards. 
%%%%%%%%%%%%%%%%%%%%%%%%%%%%%%%%%%%%%
% rethink the connect, its not as flewent for the reader to follow, 
%jumping from production processes to financial savings... but for what?
%%%%%%%%%%%%%%%%%%%%%%%%%%%%%%%%%%%%%
The study highlighted 
an average monthly cost savings of approximately \$40,000 
post-implementation, alongside notable operational 
improvements. In this case it was through the integration 
of automated \gls{kpi}s, that simplified 
the management's task of identifying areas needing 
improvement or exhibiting low performance. Comparing 
these findings with other industries, it becomes evident 
that dashboards not only enhance financial efficiency but 
also promote a proactive approach in operational 
management.



\chapter{Emerging Trends in the Dashboard Industry}
Dashboard technology is increasingly pivotal 
in transforming complex data into actionable 
insights for strategic decision-making. 
Modern dashboards emphasize interactivity 
and user-friendly designs, enabling dynamic 
data exploration and rapid comprehension. 
This chapter examines how innovations, such 
as the "Pathway Explorer," are advancing 
data visualization by offering intuitive 
interfaces and real-time processing, enhancing 
communication between analysts and 
decision-makers for more informed, 
sustainable outcomes.

The concept of interactivity in data 
exploration is increasingly recognized 
as a valuable feature, enabling users to 
dynamically engage with information. 
\autocite[p.5]{hanumanthaiahAdvancementsDataVisualization2025} 
highlights the 
benefits of interactivity, noting that 
it allows users to explore data dynamically, 
which enhances the overall user experience. 
This approach is exemplified by the 
"Pathway Explorer," a tool developed by the 
Institut de l’énergie Trottier (IET) and 
described by \autocite{levesquePathwaysExplorerInteractive2024}. 
The "Pathway Explorer" is an innovative visualization 
tool that facilitates comparisons between various 
climate transition scenarios. It provides an 
interactive platform where users can select, 
view, and dissect multiple pathways towards 
sustainability, thereby enhancing the 
decision-making process \autocite[p.32]{levesquePathwaysExplorerInteractive2024}.

The findings from Levesque et al. (\citeyear{levesquePathwaysExplorerInteractive2024}) 
further underscore the advantages of interactive 
exploration. The tool enables users to navigate 
through results with simple mouse clicks, 
eliminating the need to continuously generate 
and revise tables, thus streamlining data 
analysis \autocite[p.32]{levesquePathwaysExplorerInteractive2024}. 
Additionally, the ease of use and speed of 
the "Pathway Explorer" are significant 
benefits. Its rapid processing and visual 
interface allow users to quickly understand 
and navigate data with a minimal learning 
curve. Immediate result viewing is facilitated 
through tooltips that display data points 
upon hovering, enhancing the efficiency of 
data exploration .

Moreover, the tool supports quick access 
to comparative data measures, such as percentages, 
and facilitates easy toggling between different 
views and scenarios for more detailed analyses \autocite[p.32]{levesquePathwaysExplorerInteractive2024}. 
This feature is particularly useful for users 
seeking to perform comprehensive evaluations 
and draw meaningful insights from complex 
datasets.

In comparison, the study by \citeauthor{dalbahInteractiveDashboardPredicting2022} (\citeyear[p.1]{dalbahInteractiveDashboardPredicting2022}) 
emphasizes the importance of user-friendly 
interfaces in dashboards. These dashboards 
are designed to facilitate exploratory 
analysis of customer data and predictions 
of loyalty status. They serve as a 
communication bridge between data analysts 
and organization managers, providing actionable 
insights that can drive strategic decisions 
\autocite[p.1]{dalbahInteractiveDashboardPredicting2022}. 
While both the "Pathway Explorer" and dashboards 
aim to enhance data accessibility and usability, 
the former focuses on sustainability scenarios, 
whereas the latter centers on customer data analysis.

In evaluating these sources, it is clear that 
both emphasize the significance of interactivity 
and user-friendly design in data tools. 
However, the "Pathway Explorer" offers a 
specialized focus on climate transition scenarios, 
providing a unique platform for sustainability 
analysis, while dashboards, as discussed by 
\citeauthor{dalbahInteractiveDashboardPredicting2022}, 
cater to a broader range of 
business applications. Both approaches 
demonstrate the evolving landscape of data 
visualization tools, where interactivity 
and ease of use are paramount for effective 
data exploration and decision-making.

The integration of cloud computing and 
machine learning into dashboard technology 
represents a significant advancement in 
how organizations manage and utilize data. 
\citeauthor{bussaEnhancingBITools2023} (\citeyear{bussaEnhancingBITools2023}) 
emphasizes the transformative 
power of cloud computing, which allows for 
the storage, processing, and analysis of 
large data sets without the need for 
detailed on-premises infrastructures. 
This capability facilitates global 
collaboration, making it easier for 
organizations to work together across 
geographic boundaries \autocite[p. 82]{bussaEnhancingBITools2023}.

\citeauthor{gurcanBusinessIntelligenceStrategies2023} (\citeyear{gurcanBusinessIntelligenceStrategies2023}) 
provide a comprehensive 
evaluation of business intelligence 
strategies from 2003 to 2023, identifying 
"Organizational capability" as the fastest 
emerging trend. This trend underscores the 
importance of an organization's ability to 
assemble, integrate, and distribute resources 
effectively, turning organizational knowledge 
into tangible value \autocite[p. 14]{gurcanBusinessIntelligenceStrategies2023}. 
This insight is crucial for understanding how 
BI solutions can shape the future path of 
organizations by facilitating resource management 
and strategic planning.

In contrast, the second fastest emerging 
trend identified by \citeauthor{gurcanBusinessIntelligenceStrategies2023} 
is the application of AI in business intelligence. 
AI's ability to analyze vast amounts of data 
and provide recommendations makes analytics 
accessible to both data scientists and average 
users, thus democratizing data insights across 
the organization. This trend highlights the 
growing importance of AI in making complex data 
understandable and actionable.

\citeauthor{hamzehiBusinessIntelligenceUsing2022} (\citeyear{hamzehiBusinessIntelligenceUsing2022}), 
on the other hand, focus on the 
application of machine learning within the 
pharmaceutical industry, demonstrating how 
clustering algorithms can optimize product 
sales systems by segmenting customers based 
on purchasing behaviors \autocite[p. 33233]{hamzehiBusinessIntelligenceUsing2022}. 
Their study provides a detailed look at how 
machine learning can enhance business processes, 
offering a specific industry perspective on the 
broader trends identified by \citeauthor{gurcanBusinessIntelligenceStrategies2023}.

The exploration in the reasearch of \citeauthor{dalbahInteractiveDashboardPredicting2022} (\citeyear{dalbahInteractiveDashboardPredicting2022})
concentrates on 
machine learning algorithms in dashboards 
to predict customer attrition rates, using 
techniques such as Logistic Regression and 
Random Forest to enhance predictive capabilities 
\autocite[p.1]{dalbahInteractiveDashboardPredicting2022}. 
This study 
complements other trends by illustrating 
how machine learning can be applied to improve 
customer loyalty insights, thereby enhancing 
business performance.

In evaluating these sources, \citeauthor{bussaEnhancingBITools2023}'s work 
provides a foundational understanding of cloud 
computing's role in facilitating global 
collaboration, while \citeauthor{gurcanBusinessIntelligenceStrategies2023} 
offer a longitudinal analysis of business intelligence 
trends, providing context for the technological 
advancements discussed. Hamzehi and Hosseini's 
study adds depth by focusing on industry-specific 
applications, and \citeauthor{dalbahInteractiveDashboardPredicting2022} 
demonstrate 
practical implementations of machine learning in 
dashboards. Together, these sources paint a 
comprehensive picture of current trends in 
dashboard technology, highlighting the importance 
of integrating cloud computing and machine 
learning to drive strategic and sustainable 
business outcomes.

Following the exploration of current trends 
in dashboard technology, it is essential to 
delve into specific tools that exemplify 
these advancements. Tableau and Power BI 
are two leading platforms in the realm of 
data visualization and business intelligence, 
each offering distinct features and benefits 
that cater to different organizational needs.

Tableau is renowned for its sophisticated data 
visualization capabilities, providing users 
with a wide array of options, including bar 
charts, line charts, Gantt charts, and maps, 
which facilitate the creation of engaging visual 
representations \autocite[p.159]{beardTableauVersion202032021}. 
The platform's 
drag-and-drop interface simplifies the process 
of building visualizations, making it accessible 
to users with varying technical expertise levels. 
Additionally, Tableau's "Ask Data" feature 
allows users to pose questions in natural 
language and receive instant visualizations 
as answers, enhancing interactive data exploration. 
The ability to connect multiple data sources 
through Tableau Relationships streamlines data 
integration, particularly beneficial in contexts 
like library work. Furthermore, 
accessibility improvements ensure that users 
with disabilities can effectively engage with 
visualizations, broadening the platform's usability.

In contrast, Power BI is distinguished by its 
user-friendly interface and seamless integration 
with other Microsoft products, making it 
accessible for beginners \autocite[p.1046]{sahayaPoweringSalesInsights2024}. 
Power BI dashboards consolidate various data 
visualizations into a singular platform, 
simplifying the visualization of \gls{kpi}s \autocite[p.130]{khatuwalBusinessIntelligenceTools2022}. 
It offers visualizations such as benchmarks, 
tree charts, funnel charts, and fill charts, 
along with connectors for \gls{saas} services like 
GitHub, \gls{crm}, Salesforce, \gls{erp}, Sendgrid, and 
Zendesk.
Power BI's live connectivity to \gls{saas} services, 
without transferring data from Microsoft 
SQL Server to the cloud, enhances security 
and efficiency. 
The Power BI Designer allows users to import data, 
model it, and publish updates, providing 
flexibility in data management.

In evaluating these platforms, 
Tableau is optimal for larger data sets, 
offering advanced visualization tools that 
require tutorials and training to fully leverage 
their potential \autocite[p.1046]{sahayaPoweringSalesInsights2024}. 
Conversely, Power BI is more suited for small to 
medium-sized data sets, emphasizing simplicity 
and integration, crucial for organizations seeking 
straightforward business intelligence solutions. 
Both tools demonstrate the importance of 
tailored solutions in the evolving landscape 
of data visualization, where user accessibility 
and integration capabilities are key factors in 
enhancing organizational decision-making and 
strategic planning.
sdf
\chapter{Methodology}
\chapter{Assessment of Bayer's Current Position}
\chapter{Impact of Industry Trends on Bayer}
\chapter{Future Implications for Bayer}
\chapter{Conclusion}