% This file contains the main content of the document.
% It starts with an introduction chapter.

\chapter{Introduction}

% This is an example of how to use an acronym.
% The 'gls' command is used to insert the acronym, and the glossary package handles the definition.
\gls{erp}

% This is an example of how to include a figure.
% The 'figure' environment is used to create a floating figure.
% 'includegraphics' is used to insert the image file.
% 'caption' adds a caption to the figure, and 'label' allows for cross-referencing.
% \begin{figure}[h]
%     \centering
%     \includegraphics[trim={0 10cm 0 8cm},clip, width=\textwidth]{location}
%     \caption{Caption content}
%     \label{fig:types_of_bi_reports}
%     \begin{flushleft}
%         \small{Source: Own illustration}
%     \end{flushleft}    
% \end{figure}


% This is an example of how to create a table.
% The 'table' environment is used to create a floating table.
% 'caption' adds a caption to the table, and 'label' allows for cross-referencing.
% The 'tabular' environment is used to create the table structure.
% \begin{table}[h!]
% \caption{Overview main codes, counts and descriptions}
% \label{tab:main_codes}
% \centering
% \begin{tabular}{|p{5,5cm}|c|p{8cm}|}
% \hline
% \textbf{Category} & \textbf{Count} & \textbf{Description} \\

% \hline
% \end{tabular}

% \begin{flushleft}
%         \small{Source: Own illustration}
%     \end{flushleft} 
% \end{table}
