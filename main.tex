\documentclass[
	12pt, %Schriftgröße
	a4paper,
	listof=totoc, %Inhaltsverzeichniseinträge für Listen (z.B. Abbildungen)
	bibliography=totoc, %Inhaltsverzeichniseinträge für Quellen
	numbers=noenddot, %Entfernt Punkt hinter Gliederungsnummern
	english, %Sprache
	headseplie, %Headertrennlinie
	%footsepline, %Footertrennlinie
	oneside %einseitiges Druckformat %%% Unterdrücken der leeren Seite nach Titelblatt
	]{scrbook} %Dokumentenklasse (Koma-Script)
\usepackage[T1]{fontenc}
\usepackage{mathptmx} % Set main font to Times
\usepackage{float}
\usepackage[utf8]{inputenc}
\usepackage{babel} 
\usepackage{url}
\usepackage{xurl} %URLs in der Bibliographie
\usepackage{graphicx} %Bilder einfügen
\usepackage{caption}
\captionsetup{justification=raggedright, singlelinecheck=false}

%\usepackage{pdfpages} %PDF einfügen
\usepackage[a4paper, margin=1in]{geometry}
\usepackage[right]{eurosym} %Euro-Zeichen
\usepackage{amssymb}
\usepackage{csquotes} % Required by biblatex
\usepackage[backend=biber, style=bath, sorting=ynt]{biblatex}
\assignrefcontextentries[]{*}

% \usepackage[backend=biber, bibstyle=biblatex-bath] {biblatex}
\addbibresource{citations.bib} %Bibliographie-Datei
\DefineBibliographyStrings{english}{%
  urlseen = {Accessed},
}

\DeclareCiteCommand{\citeauthor}
    {}
    {\bibhyperref{\printnames{labelname}}}
    {\multicitedelim}
    {}

\DeclareCiteCommand{\citeyear}
    {}
    {\bibhyperref{\printfield{year}}}
    {\multicitedelim}
    {\usebibmacro{postnote}}

\DeclareCiteCommand{\citeyearpar}
    {}
    {\mkbibparens{\bibhyperref{\printfield{year}, \usebibmacro{postnote}}}}
    {\multicitedelim}
    {}

\DeclareCiteCommand{\citeyearparnopage}
    {}
    {\mkbibparens{\bibhyperref{\printfield{year}}}}
    {\multicitedelim}
    {}

\DeclareCiteCommand{\citeauthoryearpar}
    {}
    {\bibhyperref{\printnames{labelname}\space\mkbibparens{\printfield{year}, \usebibmacro{postnote}}}}
    {\multicitedelim}
    {}

\DeclareCiteCommand{\citeauthoryearparnopage}
    {}
    {\bibhyperref{\printnames{labelname}\space\mkbibparens{\printfield{year}}}}
    {\multicitedelim}
    {}


% \usepackage{tocloft} % Inhaltsverzeichnis anpassen

% Add minitoc setup (adapted from your working MWE)
% \ProvidesFile{mtc-fko.tex}%
% \usepackage[tight]{minitoc}
% \setlength{\mtcindent}{0pt}
% \mtcsetfont{minitoc}{section}{\sectfont\small}
% \mtcsettitlefont{minitoc}{\sectfont\large}
% \mtcsettitle{minitoc}{Contents} % Changed from "Inhalt" to "Contents" for English

% \setlength{\mtcindent}{24pt} 
% \renewcommand{\mtcoffset}{0pt} 
% \mtcsetoffset{minitoc}{0pt} 
% \setlength{\mtcskipamount}{\bigskipamount}

% \setcounter{minitocdepth}{2}
% \renewcommand{\mtcfont}{\small\rmfamily\upshape\mdseries} 
% \renewcommand{\mtcSfont}{\small\rmfamily\upshape\bfseries}

% % Save original section command
% \let\originalsection\section
% \let\originalsubsection\subsection

% % Flag for appendix mode
% \newif\ifappendixmode
% \appendixmodefalse

% % Redefine section to conditionally add to TOC
% \renewcommand{\section}[1]{%
%   \ifappendixmode
%     % In appendix: add to local TOC only, not main TOC
%     \originalsection*{#1}%
%     \addcontentsline{appendix}{section}{#1}%
%   \else
%     % Normal behavior: add to main TOC
%     \originalsection{#1}%
%   \fi
% }

% \renewcommand{\subsection}[1]{%
%   \ifappendixmode
%     % In appendix: add to local TOC only
%     \originalsubsection*{#1}%
%     \addcontentsline{appendix}{subsection}{#1}%
%   \else
%     % Normal behavior
%     \originalsubsection{#1}%
%   \fi
% }


% \usepackage{etoolbox}
% % Flag to control TOC behavior
% \newbool{appendixmode}

% % Patch section command to conditionally add to TOC
% \pretocmd{\section}{%
%   \ifbool{appendixmode}{%
%     \addcontentsline{toc}{section}{}% Add empty entry or skip entirely
%   }{}%
% }{}{}

\usepackage[utf8]{inputenc} % Eingabe-Encoding
% Redeclare section commands to set tocindent to 0pt
\RedeclareSectionCommand[tocindent=0pt]{chapter}
\RedeclareSectionCommand[tocindent=0pt]{section}
\RedeclareSectionCommand[tocindent=0pt]{subsection}
\RedeclareSectionCommand[tocindent=0pt]{subsubsection}
\RedeclareSectionCommand[tocindent=0pt]{paragraph}
\RedeclareSectionCommand[tocindent=0pt]{subparagraph}

% You might also want to adjust numwidth if needed:
% \RedeclareSectionCommand[tocnumwidth=2em]{chapter}
% \RedeclareSectionCommand[tocnumwidth=2em]{section}

\usepackage{setspace} % Zeilenabstand



\usepackage{enumitem}
\usepackage{fancyhdr}
\usepackage{fancyref}
\pagestyle{fancy}
\fancyhf{} % Clears all header and footer fields
\fancyhead[C]{\thepage} % Puts the page number in the center of the header
\renewcommand{\headrulewidth}{0pt} % Removes the horizontal line in the header (optional)
\RedeclareSectionCommand[style=section]{chapter} 

% Redefine the 'plain' style for pages that automatically use it (like first pages of chapters/sections)
\fancypagestyle{plain}{%
  \fancyhf{}%
  \fancyhead[C]{\thepage}%
  \renewcommand{\headrulewidth}{0pt}%
}

\usepackage{blindtext} % Lorem-Ipsum-Plugin


% \usepackage[
% 	nonumberlist, %keine Seitenzahlen anzeigen
% 	acronym,      %ein Abkürzungsverzeichnis erstellen
% 	toc,          %Einträge im Inhaltsverzeichnis
% 	section      %im Inhaltsverzeichnis auf section-Ebene erscheinen
% 	]
% {glossaries}
\usepackage[ 
   colorlinks,        % Links ohne Umrandungen in zu wählender Farbe 
   linkcolor=black,   % Farbe interner Verweise 
   filecolor=black,   % Farbe externer Verweise 
   citecolor=black,   % Farbe von Zitaten 
   urlcolor=black	  % Farbe von Links
   ]{hyperref} %Verlinkungen
\urlstyle{rm}
\usepackage[figure]{hypcap}
\usepackage[acronym, nonumberlist, ]{glossaries} %% use after hyperref %Glossar-Paket laden
\setglossarystyle{long}
\renewenvironment{theglossary}%
 {\begin{longtable}[l]{lp{\glsdescwidth}}}%
 {\end{longtable}}
 

\usepackage{listings,xcolor} %Codeanzeige
\usepackage[normalem]{ulem}
\usepackage{amsmath}
\useunder{\uline}{\ul}{}

\usepackage{array}
\usepackage{geometry}
\geometry{a4paper, margin=1in}
\usepackage{booktabs}
\usepackage{chngcntr}
\counterwithout{figure}{chapter}
\counterwithout{table}{chapter}

\definecolor{dkgreen}{rgb}{0,.6,0}
\definecolor{dkblue}{rgb}{0,0,.6}
\definecolor{dkyellow}{cmyk}{0,0,.8,.3}

\lstset{
    numbers=left, 
    numberstyle=\tiny, 
    numbersep=5pt,
    breaklines=true,
    frame=single,
    escapeinside={(*@}{@*)}, %nicht anzuzeigende Ausdrücke, z.B. für Labels
    language=sh,
    basicstyle=\ttfamily\fontsize{10}{12}\selectfont,
    keywordstyle    = \color{dkblue},
    stringstyle     = \color{red},
    identifierstyle = \color{dkgreen},
    commentstyle    = \color{gray},
    emph            =[1]{php},
    emphstyle       =[1]\color{black},
    emph            =[2]{if,and,or,else},
    emphstyle       =[2]\color{dkyellow}
    } 

  % Set the chapter title font size to 17pt
\setkomafont{chapter}{\LARGE}

%%%%%%%%%%%%%%%%%%%%%%%%%%%%%%%%%%%%%%%%%%%%%%%%%%%%%
%%%%%%%%%%% Sonderformatierung
%%%%%%%%%%%%%%%%%%%%%%%%%%%%%%%%%%%%%%%%%%%%%%%%%%%%%

% Seitenabstände definieren
\geometry{verbose,tmargin=3cm,bmargin=2cm,lmargin=2.1cm,rmargin=3cm} 

% Hurenkinder und Schusterjungen verhindern (Ja, das heißt wirklich so!!!)
\clubpenalty = 10000 \widowpenalty = 10000 \displaywidowpenalty = 10000 

\newcommand{\footfigref}[1]{\footnote{Abb. \ref{#1} auf Seite \pageref{#1}}}


% Vertikaler Abstand zwischen Ende Textblock - Ende Fußzeile --> Abstand der Seitenzahl von Rand erhöhen 
\setlength{\footskip}{10mm}

% Abstand vor/nach Überschriften verändern

\RedeclareSectionCommand[%
    beforeskip=0.5\baselineskip,
    afterskip=0.5\baselineskip
]{chapter}

\RedeclareSectionCommand[%
    beforeskip=0.5\baselineskip,
    afterskip=0.5\baselineskip
]{section}

\RedeclareSectionCommand[%
    beforeskip=0.1\baselineskip,
    afterskip=0.1\baselineskip
]{subsection}

\RedeclareSectionCommand[%
    beforeskip=0.01\baselineskip,
    %%afterskip=0.2\baselineskip
]{paragraph}

\setlength{\abovecaptionskip}{4pt}  % 1pc=12pt 
\setlength{\belowcaptionskip}{0pt}
%\setlength{\textfloatsep}{4pt}
\setlength{\intextsep}{1pc}

%% Verkleinerung der Textgröße unter Abbildungen
\addtokomafont{caption}{\small}

% Den Punkt am Ende der Glossareinträge deaktivieren
\renewcommand*{\glspostdescription}{}

%Glossar-Befehle anschalten
%\makeglossaries

% sorgt dafür, dass bei Leerzeile die Einrückung verhindert und stattdessen eine Leerzeile eingefügt wird % erspart bigskips und erhöht die Lesbarkeit im LaTeX-Text 
\KOMAoptions{parskip=full*}

% % ändert Titelschriftart in Serifen-Normalschriftart
\addtokomafont{disposition}{\rmfamily} 

\makenoidxglossaries

\loadglsentries{glossar.tex}

%%%%%%%%%%%%%%%%%%%%%%%%%%%%%%%%%%%%%%%%%%%%%%%%%%%%%
%%%%%%%%%%% Textbausteine
%%%%%%%%%%%%%%%%%%%%%%%%%%%%%%%%%%%%%%%%%%%%%%%%%%%%%
%%%%%%%%%%%% Student Name
\newcommand{\studentlastname}{-}
\newcommand{\studentfirstname}{-}
%%%%%%%%%%%% Typ der Arbeit
\newcommand{\type}{-}
%%%%%%%%%%%% Thema
\newcommand{\topic}{-}
%%%%%%%%%%%% Untertitel
\newcommand{\subtopic}{-}
%%%%%%%%%%%% Course
\newcommand{\course}{-}
%%%%%%%%%%%% Company
\newcommand{\company}{-}
%%%%%%%%%%%% Supervisor HWR
\newcommand{\supervisorhwr}{-}
%%%%%%%%%%%% Intake
\newcommand{\intake}{-}
%%%%%%%%%%%% Semester
\newcommand{\semester}{-}
%%%%%%%%%%%% Date of submission
\newcommand{\dateofsubmission}{-}
%%%%%%%%%%%% Place of submission
\newcommand{\placeofsubmission}{-}
%%%%%%%%%%%% Rund um den Sperrvermerk
\newcommand{\titleSperrvermerk}{STATEMENT OF CONFIDENTIALITY}
%%%%%%%%%%%%%%%%%%%%%%%%%%%%%%%%%%%%%%%%%%%%%%%%%%%%%>>>>>>>


%%%%%%%%%%%%%%%%%%%%%%%%%%%%%%%%%%%%%%%%%%%%%%%%%%%%%>>>>>>>
% Metadaten für das PDF
\hypersetup{%
  pdftitle={\topic},
  pdfsubject={\subtopic},
  pdfauthor={\studentfirstname~\studentlastname},
  pdfkeywords={\type, \topic, \subtopic, \course, \company, \supervisorhwr, \intake, \semester},
}
\begin{document} 

% \setcounter{maxnames}{3} % Setzt die maximale Anzahl der Autoren in der Bibliographie

% falsche Default-Silbentrennung überschreiben
% This file contains custom hyphenation rules for the document.
% The \hyphenation command is used to specify correct hyphenation points for words that LaTeX might not handle correctly by default.

\hyphenation{Soft-ware-ent-wick-lung}
\hyphenation{Be-nut-zer-freund-lich-keit}
\hyphenation{Web-app-li-ka-tion}


% Statement of Confidentiality
%%%%%%%%%%%%%%%%%%%%%%%%%%%%%%%%%%%%%%%%%%%%%%%%%%%%%>>>>>>>
%%%%%%%%%%% Statement of Confidentiality,comment, if not neccessary
\thispagestyle{empty}
\begin{center}
	\type

	\vspace{1.5cm}

	\studentfirstname~\studentlastname

	\huge \textbf{\titleSperrvermerk}
\end{center}

\begin{spacing}{1.5}
	Due to the confidential data included, the following chapters are subject to a 
non-disclosure clause and exclusively provided for the responsible 
Course Director and the appointed supervisor / examiner(s).
\end{spacing}

 {\huge \textcolor{red}{INSERT}} %Insert Chapters marked as: confidential


\vspace{2cm} % Define based on space left

\begin{spacing}{1.5}
A short version of this assignment solely containing the chapters and/or sections 
not subjectto this Statement of Confidentiality is provided in digital form under
%enter file name here
on SAM.
\end{spacing}


% Front Page
\newpage
% This file defines the layout and content of the front page.

% The page style is set to 'empty' to remove headers and footers.
\thispagestyle{empty}

% The title and other front page elements are centered.
\begin{center}
	% The main topic of the document is displayed in a huge font size.
	\huge\topic

	% The subtopic is commented out but can be enabled if needed.
	% \large\subtopic
	
	\vspace{1.5cm}
	\normalsize{
	% The type of the work (e.g., Bachelor Thesis) is displayed.
	\type

	\vspace{1.5cm}
	% The date of submission is displayed.
	presented on \dateofsubmission 

	\vspace{1.5cm}
	to

	\vspace{1.5cm}
	% The name of the university is displayed.
	Hochschule für Wirtschaft und Recht Berlin
}

\end{center}	

\vspace{1.5cm}

% The array stretch is increased to add more space between table rows.
\renewcommand{\arraystretch}{2}
% A table is used to display student and project information.
\begin{tabular}{l l}
	Name: & \studentfirstname~\studentlastname \tabularnewline
	Course: & \course \tabularnewline
	Intake: & \intake \tabularnewline
	Semester: & \semester \tabularnewline
	Company: & \company \tabularnewline
	Supervisor HWR: & \supervisorhwr \tabularnewline
\end{tabular}


\onehalfspacing % anderthalbfacher Zeilenabstand

%%%%%%%%%%%%%%%%%%%%%%%%%%%%%%%%%%%%%%%%%%%%%%%%%%%%%%%%%%%%%%%%%%%%%%%%%%%%%%%%%%%%%%%%%%%%%%%%%%%%%%%%%%%%%%%%%%%%%%%%%%%
%%%%%%%%%%% Dokumenteninhalt START
%%%%%%%%%%%%%%%%%%%%%%%%%%%%%%%%%%%%%%%%%%%%%%%%%%%%%%%%%%%%%%%%%%%%%%%%%%%%%%%%%%%%%%%%%%%%%%%%%%%%%%%%%%%%%%%%%%%%%%%%%%%



\pagenumbering{Roman} % römische Seitenzahlen

%%%%%%%%%%%%%%%%%%%%%%%%%%%%%%%%%%%%%%%%%%%%%%%%%%%%%
%%%%%%%%%%% Inhaltsverzeichnis, Tabellen, Abbildungen, etc.
\newpage
\tableofcontents{}

% Glossar und Akronyme
\addcontentsline{toc}{chapter}{Akronyms}
\textnormal{\printnoidxglossaries}

% Abbildungsverzeichnis
\newpage
\listoffigures


% Tabellenverzeichnis
\newpage
\listoftables

\clearpage

%% arabische Seitenzahlen
\pagenumbering{arabic}


%main text
\chapter{Introduction}






%%%%%%%%%%%%%%%%%%%%%%%%%%
% Quellen
%%%%%%%%%%%%%%%%%%%%%%%%%
%Literaturverzeichnis
%\addtocontents{toc}{\protect\vspace*{\baselineskip}}
% \nocite{*}
%% \bibliographystyle{alpha} %% tu es nicht, niemals, das ist eklig, nicht einkommentieren
% \setcounter{maxnames}{999}
\newpage
\newrefcontext[sorting=nyt]
\chapter{Reference List}
\vspace*{-2\baselineskip} % Abstand zum Titel verringern
\textnormal{\printbibliography[heading=none]}

%Appendix
\newpage
\chapter{Appendix}

\begin{tabular}{p{14.9cm}r}
\hyperref[app:A]{Appendix A: TBD} & \pageref{app:A} \\
\hyperref[app:B]{Appendix B: TBD} & \pageref{app:B} \\
\end{tabular}
\pagebreak

\section*{Appendix A:}
\label{app:A} 

\section*{Appendix B:}
\label{app:B}


%Statement of Academic Integrity
\newpage
% This file contains the statement of academic integrity.

% The statement begins with a chapter heading.
\chapter{Statement of Academic Integrity}

% The main text of the declaration.
I solemnly declare that I have independently completed the present work in all its parts and
have not used any sources or aids other than those specified in the work. Furthermore, this
work has not been submitted in the same or similar form for any other examination. All
verbatim or paraphrased quotations, as well as all sections that were designed, drafted,
and/or edited with the help of AI-based tools, are clearly marked and documented. In the
appendix of my work, I have listed all AI-based tools used, along with product names and
formulated inputs (prompts) in an AI directory.

I am aware that the use of texts or other content and products generated by AI-based tools
does not guarantee their quality. I take full responsibility for any machine-generated
passages I have used and bear the responsibility for any potentially erroneous or distorted
content generated by the AI, incorrect references, violations of data protection and copyright
laws, or plagiarism.

% The signature section is centered.
\begin{center}
    % A table is used to format the signature line.
    \begin{tabular}{lp{4em}l} % 'l' for left, 'p{4em}' for fixed width, 'r' for right
 \rule{0pt}{2cm} % Invisible rule to ensure some height for the image row
 \placeofsubmission, \dateofsubmission \hspace{6cm} & & %\includegraphics[width=3cm]{} -> Enter the picture of your signature here
  \hspace{3cm}    \hspace{3cm} \\ \cline{1-1}\cline{3-3}
 Place, Date & & \studentfirstname~\studentlastname
\end{tabular}
\end{center}



\end{document}

